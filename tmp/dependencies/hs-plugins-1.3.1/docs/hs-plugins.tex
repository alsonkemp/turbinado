\documentclass{article}

\usepackage{url}
\usepackage{tex2page}

% typeset math as ascii
\htmlmathstyle{no-in-text-image no-display-image}

% something other than |
\verbescapechar\&

\cssblock 
h1      {font-size: 16pt}
h2      {font-size: 15pt}
\endcssblock 

% color of verbatim elements
\cssblock
.verbatim {color: grey20}

.scheme  .variable  {color: grey20} 
.scheme  .keyword   {color: navy}
.scheme  .builtin   {color: maroon}
\endcssblock

% add some extra keywords
\scmkeyword{as case class data default deriving do else hiding if}
\scmkeyword{import in infix infixl infixr instance let module newtype}
\scmkeyword{of qualified then type where forall \\ }
\scmbuiltin{: :: = -> <- @ ~ => - >>= >> }

\newcommand{\code}[1]{{\texttt{#1}}}
\newcommand{\hsplugins}{{\texttt{hs-plugins}}}

\title{hs-plugins\\
       Dynamically Loaded Haskell Modules}

\author{\urlh{http://www.cse.unsw.edu.au/~dons}{Don Stewart}}

\begin{document}

\maketitle

\medskip
%
{\htmlonly \textbf{Download \endhtmlonly 
\urlh{ftp://ftp.cse.unsw.edu.au/pub/users/dons/hs-plugins/hs-plugins-1.0.tar.gz}
     {version 1.0}}
%
\medskip

\hsplugins{} is a library for loading code written in Haskell into an
application at runtime, in the form of plugins. It also provides a
mechanism for (re-)compiling Haskell source at runtime. Thirdly, a
combination of runtime compilation and dynamic loading provides a set of
\code{eval} functions-- a form of runtime metaprogramming. Values
exported by Haskell plugins are transparently available to Haskell host
applications, and bindings exist to use Haskell dynamically in C, Perl
and Objective C programs. \hsplugins{} requires GHC 6.4 or later.

\medskip

% grr. double spaced.

\tableofcontents

\newpage

\section{Download}

\begin{itemize}

\item
Download the latest stable release:\\
\url{ftp://ftp.cse.unsw.edu.au/pub/users/dons/hs-plugins/hs-plugins-1.0.tar.gz}

\item
Darcs repository of the latest code:\\
\url{darcs get http://www.cse.unsw.edu.au/~dons/code/hs-plugins}

\item
A tarball of the document you are reading:\\
\url{http://www.cse.unsw.edu.au/~dons/hs-plugins/hs-plugins.html.tar.gz}

\item
A postscript version of the document you are reading:\\
\url{http://www.cse.unsw.edu.au/~dons/hs-plugins/hs-plugins.ps.gz}

\item
Papers:
\begin{itemize}
\item A paper on interesting uses of \hsplugins{} to enable Haskell to be used
as an application extension language:\\
\url{http://www.cse.unsw.edu.au/~dons/papers/PSSC04.html}

\item
A paper on dynamic applications in Haskell, utilizing \hsplugins{}:\\
\url{http://www.cse.unsw.edu.au/~dons/papers/SC05.html}
\end{itemize}

\end{itemize}

It is known to run on \code{i386-\{linux,freebsd,openbsd\}},
\code{sparc-solaris2}, \code{powerpc-\{macosx,linux\}} and flavours of
Windows.

\section{History}

\begin{itemize}
        \item Jan 2006, v1.0
        \begin{itemize}
        \item Cabalised build system
        \item Behave better on 64 bit platforms
        \item Haddock documentation of the API.
        \item A couple of bug fixes
        \end{itemize}

        \item June 2005, v0.9.10
        \begin{itemize}
        \item Support for GHC 6.4, with help from Sean
                        Seefried for the package.conf parser.
        \item Ported to Windows of various flavours thanks to Vivian McPhail and Shelarcy
        \item Removed posix and unix dependencies
        \item Now uses HSX parser, thanks to Niklas Broberg
        \item Extended load interface, thanks to Lemmih
        \item Source now in a darcs repository
        \item Supports building with GNU make -jN
        \item Simplified module hierarchy, moved under System.* namespace
        \item pdynload clarifications, thanks to Alistair Bayley 
        \item Miscellaneous bug fixes
        \end{itemize}

	\item February 2005, v0.9.8
        \begin{itemize}
		\item Fix bug in .hi parsing. 
		\item Add reloading of packages. 
		\item Fix bug in canonical module names 
                      (fixing problems with "Foo.o" and "./Foo.o"
		\item Fix for hierarchical names, don't guess them,
			read them from the .hi file.
		\item Add new varients of load.
		\item Fix bug in makeAll, such that dependent module
			changes were not noticed.
		\item Add varient of eval:\code{ unsafeEval\_}, returing
			Either.
		\item Better, bigger testsuite.
		\item Better api.
	\end{itemize}

	\item September 2004.
        \begin{itemize}
		\item makeAll
		\item Better return type for make.
	\end{itemize}

	\item Mid August 2004, v0.9.6 release.
        \begin{itemize}
		\item More portable, thanks to debugging by Niklas Broberg.
		\item Other small fixes to the interfaces.
		\item Provides a runtime-generated printf.
	\end{itemize}

	\item Mid July 2004, added new pdynload strategy.

	\item Mid-June 2004, v0.9.5 release.
        \begin{itemize}
		\item dynamic typing is working

		\item static typing of interfaces is working

		\item Adds \code{eval}, and \code{hs\_eval}

		\item bugs fixed.
        \end{itemize}

        \item Early-June 2004, v0.9.4 release. 
        \begin{itemize}

                \item Adds a .hi file parser. We use this to work out
                plugin dependencies directly, meaning no more
                \code{.dep} files or \code{ghcp}. 

                \item It also adds a package.conf parser, meaning we
                can properly handle packages that either aren't stored
                in the normal location, don't have a canonical name,
                or are found using a -package-conf argument. Thanks to
                Sean for this work. 

                \item the interface to load() has changed to allow a
                list of package.conf files to search for packages.

                \item the interace to make() has changed, so that you
                can get back any stderr output produced during plugin
                compilation.

                \item It solves a bug whereby a package that is
                required by another package would not be loaded unless
                the plugin itself depended on this indirect package.

                \item more stable, more examples.
        \end{itemize}

        \item May 2004, v0.9.3 released, adding support for dependency
        conflict resolution between multiple plugins.  Several plugins with
        shared dependencies can now be safely loaded at once. --prefix is now
        respected in ./configure.  Thanks to Sean for this patch.

        \item v0.9.2 change licence to LGPL

        \item v0.9.1 expand on the documentation

        \item v0.9   released, initial source release

\end{itemize}

\section{Acknowledgements}

\begin{itemize}

\item Andr\'e Pang's \code{runtime\_loader} was the inspiration and basis
of the dynamic loader (\url{http://www.algorithm.com.au}).
\hsplugins{} has benefited from many discussions with him,
particularly to do with dependency checking and dynamic typing, and
bug reports. Andr\'e wrote an objective C binding to hs-plugins, and
helped with the design of eval(). He also fixed GHC so we could load
the dynamic loader dynamically.

\item Sean Seefried (\url{http://www.cse.unsw.edu.au/~sseefried}) was
the first user of \hsplugins{} and his code and feedback have helped
make the library much more useful and powerful.

\item Manuel Chakravarty's \code{take} system provided the basis for
\code{make}, and helped with several issues to do with safety of
plugins, apis and the applications that use them.
Manuel also helped with the design of eval(), and on how to
successfully evaluate polymorphic functions using rank-N types.

\item Simon Marlow helped with several issues to do with linking and
loading static and dynamic code, and provided many useful suggestions.

\item Hampus Ram's dynamic loader
(\url{http://www.dtek.chalmers.se/~d00ram/dynamic/}) provided the
design of the state maintained by the loader, and for thread safety
issues relating to this.

\item Shae Erisson provided several insights into more powerful uses
of the library. Thanks to everyone on \#haskell who provided
discussion about the library.

\item Malcolm Wallace's \code{hmake} provided some useful insights in
how to compile Haskell source in a way that makes it appear like an
interpreter, used in the interactive environment: \code{plugs}.

\item Niklas Broberg helped a lot by testing, and providing feedback for
the new make and load API. Thanks Niklas.

\item Finally, thanks to everyone who has worked on GHC and its
libraries: for GHCi, the .hi interface parser, the package system, and
all the other code the \hsplugins{} depends on.

\end{itemize}

\newpage

\section{Overview}

\hsplugins{} is a library for compiling and loading Haskell code into a
program at runtime. It allows you to write Haskell code (which may
be spread over multiple modules), and have an application (implemented in
any language with a Haskell FFI binding, including Haskell) load and use
your code at runtime.

\hsplugins{} provides 3 major features:
%
\begin{itemize}
        \item a dynamic loader,
        \item a compilation manager, and
        \item a Haskell evaluator
\end{itemize}

The \emph{dynamic loader} loads objects into the address space of an
application, along with any dependencies the plugin may have. The
loader is a binding to the GHC runtime system's dynamic linker, which
does single object loading. GHC also performs the necessary linking of
new objects into the running process. On top of the GHC loader is a
Haskell layer that arranges for module and package dependencies to be
found prior to loading individual modules.

The \emph{compilation manager} is a \code{make}-like system for
compiling Haskell source code into a form suitable for loading.  While
plugins are normally thought of as strictly object code, there are a
variety of scenarios where it is desirable to be able to inspect the
source code of a plugin, or to be able to recompile a plugin at runtime.
The compilation manager fills this role. It is particularly useful in
the implementation of \code{eval}.

The \emph{evaluator}, \code{eval}, utilizes the loader and compilation
manager. When passed a string containing a Haskell expression, it
compiles the string to object code, loads the result, and returns a
Haskell value representing the compiled string to the caller. It can be
considered a Haskell interpreter, implemented as a library, and can be
used to embed Haskell evaluation facilities in an application.

\section{Dynamic Loader}

The interface to the \hsplugins{} library can be divided into a number
of sections representing the functional units of the library.
Additionally, depending on the level of trust the application places
in the plugins, a variety of additional checks can be made on the
plugin as it is loaded. The levels of type safety possible are
summarised at the end of Section \ref{sec:compilation-manger} section.
The dynamic loader is available by using \code{-package plugins}.

\subsection*{Interface}
%
\begin{quote}
\scm{
import System.Plugins

load :: FilePath 
     -> [FilePath] 
     -> [PackageConf] 
     -> Symbol 
     -> IO (LoadStatus a)
}

\scm{
load_ :: FilePath 
      -> [FilePath] 
      -> Symbol 
      -> IO (LoadStatus a)
}

\scm{
data LoadStatus a
    = LoadSuccess Module a 
    | LoadFailure Errors
}
\end{quote}
%
Example:
%
\begin{quote}
\scm{
do mv <- load "Plugin.o" ["api"] [] "resource"
   case mv of
	LoadFailure msg -> print msg
	LoadSuccess _ v -> return v
}
\end{quote}
%
This is the basic interface to the dynamic loader. Load the object file
specified by the first argument into the address space (the library will
preload any module or package dependencies). The second argument is an
include path to any additional objects to load (possibly the API of the
plugin). The third argument is a list of paths to any user-defined
\code{package.conf} files, specifying packages unknown to the GHC
package system. \code{Symbol} is a string specifying the symbol name you
wish to lookup. \code{load} returns a \code{LoadStatus} value representing
failure, or an abstract representation of the module (for calls to
\code{unload} or \code{reload}) with the symbol as a Haskell value. The
value returned must be given an explicit type signature, or provided
with appropriate type constraints such that GHC can determine the
expected type returned by \code{load}.

\code{load\_} is provided for the common situation where no user-defined
package.conf files are required.

\begin{quote}
\scm{
dynload :: Typeable a    
        => FilePath  
        -> [FilePath]
        -> [PackageConf]
        -> Symbol
        -> IO (LoadStatus a)
}
\end{quote}
%
Example:
%
\begin{quote}
\scm{
do mv <- dynload "Plugin.o" ["api"] ["plugins.conf.inplace"] "resource"
   case mv of
	LoadFailure msg -> print msg
	LoadSuccess _ v -> putStrLn v
}
\end{quote}
%
\code{dynload} is a safer form of \code{load}. It uses dynamic types
to perform a check on the value returned by \code{load} at runtime, to
ensure that it has the type the application expects it to have.

In order to use \code{dynload}, the symbol the plugin exports must be
of type \code{AltData.Dynamic:Dynamic}. (See the \code{AltData} library
distributed with \hsplugins{}, and the \hsplugins{}
\code{examples/dynload} directory. References to \code{Typeable} and
\code{Dynamic} refer to the \hsplugins{} reimplementation of these
libraries. \code{AltData.Dynamic} is used at the moment, as there is a
limitation in the existing Data.Dynamic library in the presence of
dynamic loading).

The value wrapped up in the \code{Dynamic} must be an instance of
\code{AltData.Typeable}. If the value exported by the plugin \emph{is}
of type \code{Dynamic}, and the value wrapped by the \code{Dynamic}
does not match the type expected of it by the application,
\code{dynload} will return \code{Nothing}, indicating that the plugin
is not typesafe with respect to the application. If the value passes
the typecheck, \code{dynload} will return \code{LoadSuccess}. If the value
exported by the plugin is \emph{not} of type \code{Dynamic},
\code{dynload} will crash---this is a limitation of the existing
\code{Dynamic} library, it can only type-check \code{Dynamic} values.
Additionally, Data.Dynamic is limited to monomorphic types, or must be
wrapped inside a rank-N type to hide the polymorphism from the
typechecker. This is a bit cumbersome. An alternative typesafe
\code{load} is available via the \code{pdynload} interface, which is
able to enforce the type of the plugin using GHC's type inference
mechanism, and is not restricted in its expressiveness (at the cost of
greater load times):

\begin{quote}
\scm{
pdynload :: FilePath
         -> [FilePath]
         -> [PackageConf]
         -> Type
         -> Symbol
         -> IO (LoadStatus a)
}

\scm{
pdynload_ :: FilePath
          -> [FilePath]
          -> [PackageConf]
          -> [Arg]
          -> Type
          -> Symbol
          -> IO (LoadStatus a)
}
\end{quote}
%
Example:
%
\begin{quote}
\scm{
do v <- pdynload "Plugin.o" ["api"] [] "API.Interface" "resource"
   case v of
	LoadSuccess _ a  -> putStrLn "yay!"
	_                -> putStrLn "type error"
}
\end{quote}
%
\code{pdynload} is a replacement for \code{dynload}, which provides a
solution to the various problems caused by the existing dynamics
library in Haskell. Rather than use normal dynamics, which constrain
us to monomorphic types only (or rank-N types), it instead uses GHC's
type inference to unify the plugin's export value with that provided
by the api (via its .hi file).  It is a form of \emph{staged type inference}
for module interfaces, allowing plugins to use any type definable in Haskell.
\code{pdynload} is like \code{dynload}, but requires a new \code{Type}
argument. This can be considered a type annotation on the value the plugin
should be constrained to.

Prior to loading the object, \code{pdynload} generates a tiny Haskell
source file containing, for example:
%
\begin{quote}
\scm{
module APITypeConstraint where
import qualified API
import qualified Plugin

_ = Plugin.resource :: API.Interface
}
\end{quote}
%
It then calls GHC's type checker on this file, which runs the full
Haskell type inference machinery. If the file typecheckes, then the
plugin type is correct, and the plugin is safe to load, otherwise it
is an error. 

Because we use the full Haskell type checker, we can have a form of
dynamic typechecking, on any type expressable in Haskell. A plugin's
value may, for example, have class constraints -- something not
checkable using the standard Dyanmic type. The cost is that
\code{pdynload} is roughly 46\% slower than an unchecked load.
 
The type of the plugin's resource field must be equivalent to the
\code{Type}. There are some restrictions on the arguments that may be
passed to pdynload. Currently, we require:
        \begin{itemize}
        \item The object name has the suffix (.o) removed and this
        becomes a qualified module name in the generated type-checker
        input file.

        \item The type name must be a single fully-qualified
        type-identifier, as the module name is stripped off (i.e.
        everything up to the last ".") and used as a qualified import.
        This means that you can't use, for example, \code{"Int ->
        String"} as a type (type synonyms are fine, though).

\end{itemize}

For example, \code{pdynload "API2.o" ["./"] [] "API.PluginAPI"
"doAction"} generates:

\begin{quote}
\scm{
module <temp-generated-name> where
import qualified API    -- comes from API.PluginAPI argument
import qualified API2   -- comes from API2.o argument
_ = API2.doAction :: API.PluginAPI
}
\end{quote}

\begin{quote}
\scm{
unload :: Module -> IO ()
}

\scm{
unloadAll :: Module -> IO ()
}
\end{quote}

Unload an object, \emph{but not its dependencies} from the address
space. \code{unloadAll} performs cascading unloading of a module
\emph{and} its dependencies.

\begin{quote}
\scm{
reload :: Module -> Symbol -> IO (LoadStatus a)
}
\end{quote}

Unload, and then reload a module that must have been previously
loaded.  Doesn't reload the dependencies. \code{reload} is useful in
conjunction with \code{make}---a call to \code{reload} can be
performed if \code{make} has recompiled the plugin source.

Additionally, some support is provided to manipulation of 
libraries of Haskell modules (usually known as packages):

\begin{quote}
\scm{
loadPackage     :: String -> IO ()

unloadPackage   :: String -> IO ()

loadPackageWith :: String -> [PackageConf] -> IO ()
}
\end{quote}

\code{loadPackage} explcitly pulls in a library (which must be visible
in the current package namespace. \code{unloadPackage} unloads it.
\code{loadPackageWith} behaves like \code{loadPackage}, but you are able
to supply extra package.confs to augment the library search path.

Examples:
\begin{quote}
\scm{
do loadPackageWith "yi" ["yi.conf"]
   unloadPackage "yi-0.1"
}
\end{quote}

\newpage

\section{Compilation Manager}

The compilation manager is the system by which Haskell source code is
compiled to object code suitable for loading.

\subsection*{Interface}

\begin{quote}
\scm{
import System.Plugins

make :: FilePath 
     -> [Arg] 
     -> IO MakeStatus

makeAll :: FilePath 
        -> [Arg] 
        -> IO MakeStatus

recompileAll :: Module 
             -> [Arg] 
             -> IO MakeStatus

data MakeStatus 
        = MakeSuccess MakeCode FilePath 
        | MakeFailure Errors

data MakeCode = ReComp | NotReq
}
\end{quote}

Compile a Haskell source file to an object file, with any arguments
specified in the argument list passed through to GHC.  Returns the
build status.

\code{make} generates a GHC \code{.hi} file containing a list of
package and objects that the source depends on. Subsequent calls to
\code{load} will use this interface file to load module and library
dependencies prior to loading the object itself. \code{makeAll} also
recursively compiles any dependencies it can find using GHC's
\code{--make} flag.

\code{recompileAll} is like \code{makeAll}, but rather than relying on
\code{ghc --make}, we explicitly check a module's dependencies.

\begin{quote}
\scm{
merge   :: FilePath -> FilePath -> IO MergeStatus

mergeTo :: FilePath -> FilePath -> FilePath -> IO MergeStatus

mergeToDir :: FilePath -> FilePath -> FilePath -> IO MergeStatus

data MergeStatus 
        = MergeSuccess MergeCode Args FilePath 
        | MergeFailure Errors

type MergeCode = MakeCode
}
\end{quote}

The merging operation is extremely useful for providing extra default
syntax. An EDSL user then need not worry about declaring module names,
or having required imports.  In this way, the stub file can also be
used to provide syntax declarations that would be inconvenient to
require of the plugin author. \code{merge} will include any import and
export declarations written in the stub, as well as any module name,
so that plugin author's need not worry about this compulsory syntax.
Additionally, if a plugin requires some non-standard library, which
must be provided as a \code{-package} flag to GHC, they may specify
this using the non-standard \code{GLOBALOPTIONS} pragma.  Options
specified in the source this way will be added to the command line.
This is useful for users who wish to use GHC flags that cannot be
specified using the conventional \code{OPTIONS} pragma. The merging
operation uses the HSX parser library to parse Haskell source files.

\code{mergeTo} behaves like \code{merge}, but we can specify the file in
which to place output. \code{mergeToDir} lets you specify a directory in
which to place merged files.

\begin{quote}
\scm{
makeWith :: FilePath
         -> FilePath
         -> [Arg]
         -> IO MakeStatus
}
\end{quote}

This is a variety of \code{make} that first calls \code{merge} to
combine the plugin source with a syntax stub. The result is then
compiled. This is the preferred interface to EDSL authors who wish to
add extra syntax to a user's source. It is important to note that the
module and types from the second file argument are used to override
any of those that appear in the first argument. For example, consider
the following source files:

\begin{quote}
\scm{
module A where

a :: Integer
a = 1
}
\end{quote}

\begin{quote}
\scm{
module B where

a :: Int
}
\end{quote}

Calling \code{makeWith "A" "B" []} will merge the module name and types
from module B into module A, generating a third file:

\begin{quote}
\scm{
{-# LINE 1 "A.hs" #-}
module MxYz123 where

{-# LINE 3 "B.hs" #-}
a :: Int
{-# LINE 4 "A.hs" #-}
a = 1
}
\end{quote}

Leading to the desired result that we can ignore user-supplied module
names and types. Knowing the module name, in particular, is important
for dynamic loading, which requires the module name be known when
searching for symbols.

\begin{quote}
\scm{
hasChanged :: Module -> IO Bool
}
\end{quote}

\code{hasChanged} returns \code{True} if the module or any of its
dependencies have older object files than source files. Defaults to
\code{True} if some files couldn't be located.

\subsection*{Levels of Safety}

The normal dynamic loader, using \code{load} on object files only,
places full trust in the author of the plugin to provide a type-safe
object file, containing valid code. This can be mitigated somewhat via
the use of \code{make} to ensure that the plugin is at least Haskell
code that is well-typed internally (if we trust GHC to compile it
correctly). 

If we trust the user to provide an interface of \code{Dynamic} type, we
can check the plugin type at runtime, but the plugin's value must be
\code{Typeable}, which restricts it to be a monomorphic type (or to
using rank-N tricks). 

The greatest safety can be obtained by using \code{pdynload}, at the
cost of increased load times. \code{pdynload} essentially performs full
type inference on the plugin interface at runtime. The type safety of
the plugin, using \code{pdynload}, is then as safe as if the plugin was
statically compiled into the application. It does not provide any
\emph{further} safety than exists in static compilation. For example, it
does not preclude the use of (evil) \code{unsafeCoerce\#} to defeat
type-checking, either statically or at runtime. An extensive discussion
of type safe plugin loading is available in the \hsplugins{} paper
\urlh{http://www.cse.unsw.edu.au/~dons/hs-plugins/paper}{here}.

\newpage

\section{Eval.Haskell}

\code{eval}, and its siblings, provide a mechanism to compile and run
Haskell code at runtime, in the form of a String. The general framework
is that the string is used to create a plugin source file, which is
compiled and loaded, and type checked against its use. The resulting
value is returned to the caller. It resembles a runtime metaprogramming
\code{run} operator for closed code fragments.

\subsection*{Interface}

\begin{quote}
\scm{
import System.Eval.Haskell

eval :: Typeable a => String -> [Import] -> IO (Maybe a)

eval_ :: Typeable a =>
         String      -- code to compile
      -> [Import]    -- any imports
      -> [String]    -- extra ghc flags
      -> [FilePath]  -- extra package.conf files
      -> [FilePath]  -- include search paths
      -> IO (Either [String] (Maybe a))
}
\end{quote}

\code{eval} takes a string, and a list of import module names, and
returns a \code{Maybe} value. \code{Nothing} means the code did not
compile, or did not typecheck at its splice point. \code{Just v} gives
you \code{v}, the result of evaluating your code. It is interesting to
note that \code{eval} has the type of an interpreter. The
\code{Typeable} constraint is used to type check the evaluated code when
it is loaded, using \code{dynload}.  As usual, \code{eval\_} is a
version of \code{eval} that lets you pass extra flags to ghc and to the
dynamic loader.

The existing \code{Data.Dynamic} library requires that only monomorphic
values are \code{Typeable}, so in order to evaluate polymorphic
functions you need to wrap them up using rank-N types. Some examples:
%
\begin{quote}
\scm{
import System.Eval.Haskell

main = do i <- eval "1 + 6 :: Int" [] :: IO (Maybe Int)
          if isJust i then putStrLn (show (fromJust i)) else return ()
}
\end{quote}
 
When executed this program calls \code{eval} to compile and load the
simple arithmetic expression, returning the result, which is
displayed. If the value loaded is not of type \code{Int},
\code{dynload} will throw an exception.
 
The following example, due to Manuel Chakravarty, shows how to
evaluate a polymorphic function. Polymorphic values are not easily
made dynamically typeable, but this example shows how to do it. The
module \code{Poly} is imported as the second argument, providing the
type of the polymorphic function:
%
\begin{quote}
\scm{
import Poly
import System.Eval.Haskell

main = do m_f <- eval "Fn (\\x y -> x == y)" ["Poly"]
          when (isJust m_f) $ do
                let (Fn f) = fromJust m_f
                putStrLn $ show (f True True)
                putStrLn $ show (f 1 2)
}
\end{quote}
% 
And the type of \code{Fn}:
%
\begin{quote}
\scm{
{-# OPTIONS -fglasgow-exts #-} 
module Poly where

import AltData.Typeable

data Fn = Fn {fn :: forall t. Eq t => t -> t -> Bool}

instance Typeable Fn where
    typeOf _ = mkAppTy (mkTyCon "Poly.Fn") []
}
\end{quote}
% 
When executed, this program produces:
%
\begin{quote}
\begin{verbatim}
$ ./a.out
True
False
\end{verbatim}
\end{quote}

We thus get dynamically typeable polymorphic functions.

\begin{quote}
\scm{
unsafeEval :: String -> [Import] -> IO (Maybe a)

unsafeEval_ :: String
            -> [Import]
            -> [String]
            -> [FilePath]
            -> IO (Either [String] a)
}
\end{quote}

Wrapping up polymorphic values can be annoying, so we provide a
\code{unsafeEval} function for people who like to live on the edge,
which dispenses with dynamic typing, relying instead on the
application to provide the correct type annotation on the call to
\code{eval}. If the type loaded by \code{eval} is wrong,
\code{unsafeEval} will crash. However, its lets us remove some
restrictions on what types can be evaluated, which can be useful.

{unsafeEval\_} lets the application have full control over the import
environment and load flags to the eval call, which is useful for
applications that wish to script themselves, and require specific
modules and packages to be in scope in the eval-generated module.

This example maps a \code{toUpper} over a list:
%
\begin{quote}
\scm{
import Eval.Haskell

main = do s <- unsafeEval "map toUpper \"haskell\"" ["Data.Char"]
          when (isJust s) $ putStrLn (fromJust s)
}
\end{quote}
 
And here we evaluate a lambda abstraction, applying the result to
construct a tuple. Note the type information that must be supplied in
order for Haskell to type the usage of \code{fn}:
%
\begin{quote}
\scm{
import System.Eval.Haskell

main = do fn <- unsafeEval "(\\(x::Int) -> (x,x))" [] :: IO (Maybe (Int -> (Int,Int)))
          when (isJust fn) $ putStrLn $ show $ (fromJust fn) 7
}
\end{quote}
\subsection{Utilities for use with eval}
\code{hs-plugins} proves the following utilities for use with \code{eval}:
\begin{itemize}
\item
\code{mkHsValues} is a helper function for converting \code{Data.Map}s
of names and values into Haskell code. It relies on the assumption
that the passed values' Show instances produce valid Haskell
literals (this is true for all prelude types). It's type is as follows:
\begin{quote}
\scm{
mkHsValues :: (Show a) => Data.Map String a -> String
}
\end{quote}
\end{itemize}
% \subsection{Foreign Eval}
% 
% A preliminary binding to \code{eval} has been implemented to allow C
% (and Objective C) programs access to the evaluator. Foreign bindings
% to the compilation manager and dynamic loader are yet to be
% implemented, but shouldn't be too hard. An foreign binding to a
% Haskell module that wraps up calls to \code{make} and \code{load}
% would be fairly trivial.
%  
% At the moment we have an ad-hoc binding to \code{eval}, so that C
% programmers who know the type of value that will be returned by
% Haskell can call the appropriate hook into the evaluator. If they get
% the type wrong, a nullPtr will be returned (so calling Haskell is
% still typesafe). The foreign bindings to \code{eval} all return
% \code{NULL} if an error occurred, otherwise a pointer to the value is
% returned.
% 
% \begin{quote}
% \scm{
% foreign export ccall hs_eval_b :: CString -> IO (Ptr CInt)
% 
% foreign export ccall hs_eval_c :: CString -> IO (Ptr CChar)
%   
% foreign export ccall hs_eval_i :: CString -> IO (Ptr CInt)
% 
% foreign export ccall hs_eval_s :: CString -> IO CString
% }
% \end{quote}
% 
% An example C program for compiling and evaluating Haskell code at
% runtime follows. This program calculates a fibonacci number, returning
% it as a \code{CString} to the C program:
% %
% \begin{quote}
% \begin{verbatim}
% #include "EvalHaskell.h"
% #include <stdio.h>
% 
% int main(int argc, char *argv[])
% {
%   char *p;
%   hs_init(&argc, &argv);
%   p = hs_eval_s("show $ let fibs = 1:1:zipWith (+) fibs (tail fibs) in fibs !! 20");
%   if (p != NULL)
%      printf("%s\n",p);
%   else
%      printf("error in code\n");
%   hs_exit();
%   return 0;
% }
% \end{verbatim}
% \end{quote}
% 

\section{RTS Binding}

The low level interface is the binding to GHC's Linker.c. Therefore,
\hsplugins{} only works on platforms with a working GHCi. This library
is based on code from Andr� Pang's runtime loader. The low level
interface is as follows:

\begin{itemize}
      \item \code{initLinker}  \em  start the linker up 
      \item \code{loadObject}  \em  load a vanilla .o
      \item \code{loadPackage} \em  load a GHC library and its cbits
      \item \code{loadShared } \em  load a .so object file
      \item \code{resolveObjs} \em  and resolve symbols
\end{itemize}

Additionally, \code{Hi.Parser} provides an interface to a GHC
\code{.hi} file parser. Currently we only parse just the dependency
information, import and export information from \code{.hi} files, but
all the code is there for an application to extract other information
from \code{.hi} files.

\newpage

\section{Dynamic Loader Implementation}

The dynamic loader is the system by which modules, and their
dependencies can be loaded, unloaded or reloaded at runtime, and
through which we access the functions we need.

At its lowest level, the \hsplugins{} loader is a binding to the GHC
runtime loader and linker. This layer is a direct reimplementation of
Andre Pang's \code{runtime\_loader} (barely any code changed). The
code at this level can only load single modules, or packages/archives
(which are just objects too). Any dependency resolution must be
performed by hand.

On top of Andre's interface is a more convenient interface through
which user's should interact with the dynamic loader.  The most
significant extension to Andre's work is the automatic calculation and
loading of a plugin's package or module dependencies via .hi file
information.  It also handles initialisation of the loader, and
retrieval of values from the plugin in a more convenient way. Some
state is also stored in the loader to keep track of which modules and
packages have been loaded, to prevent unnecessary (actually, fatal)
loading of object files and packages already loaded.  Thus you can
safely load several plugins at once, that share common dependencies,
without worrying about the dependencies being loaded multiple times.
We also store package.conf information in the state, so we can work
out where a package lives and what it depends on.

The ability to remember which packages and objects have been loaded is
based on ideas in Hampus Ram's dynamic loader, which has a more
advanced dependency tracking system, with the ability to unload the
dependencies of a plugin. \hsplugins{} doesn't provide ``cascading
unloading''. The advantage \hsplugins{} has over Hampus' loader seems
to be the automatic dependency resolution via vanilla .hi files and
the dynamic recompilation stuff.

Using \code{load}, any library packages, or any \code{.o} files, that a
plugin depends upon will be automatically loaded prior to loading the
module itself. \code{load} then looks up a symbol from the object file,
and returns the value associated with the symbol as a conventional
Haskell value. It should also be possible to load a GHCi-style \code{.o}
archive of object files this way, although there is currently no way
to extract multple plugin interfaces from a archive of objects.

The application writer is not required to recalculate dependencies if
the plugin changes, and the plugin author does not need to specify
what dependencies exist, as is required in the lower level interface.
This is achieved by using the dependency information calculated by GHC
itself, stored in .hi files, to work out which modules and packages to
load, and in what order. A plugin in \hsplugins{} is really a pair of
an object file (or archive) and a \code{.hi} file, containing package
and module dependency information.  

The \code{.hi} file is created by GHC when the plugin is compiled,
either by hand or via \code{make}.  \code{load} uses a binary parser to
extract the relevant information from the \code{.hi} data. Because the
dependency information is stored in a separate file to the application
that loads the plugin, such information can be recalculated without
having to modify the application. Becaues of this, it was easy to
extend the load to support recompilation of module source, even if
dependencies change, because dependencies are no longer hard-coded
into the application source itself, but are specified by the plugin.

Assuming we have a plugin exporting some data, ``resource'', with a
record name \code{field :: String}, here is an example call to \code{load}:
%
\begin{quote}
\scm{
do m_v   <- load "Test.o" ["."] [] "resource"
   v <- case m_v of
          LoadSuccess _ v -> return v
          _               -> error "load failed"
   putStrLn $ field v
}
\end{quote}

This loads the object file \code{Test.o}, and any packages or objects
\code{Test.o} depends on. It resolves undefined symbols, and returns
from the object file the Haskell value named ``resource'', as the
value ``v''. This must be a value exported by the plugin. We then
retrieve the \code{field} component of \code{v}, and print it out.

This simple usage assumes that the plugin to load is in the same
directory as the application, and that the api defining the interface
between plugin and application is also in the current directory (hence
the ``.'' in the 2nd argument to \code{load}).

\subsection*{Dynamically Loading the Dynamic Loader}

It is also possible to load the \code{plugins} or \code{eval}
libraries in GHC. Here, for example, we load the \code{plugs}
interactive environment in GHCi, and evaluated some code. The source to
\code{plugs} is in Appendix
\ref{sec:plugs}.
%
\begin{quote}
\begin{verbatim}
paprika$ ghci -package-conf ../../../plugins.conf.inplace -package eval
   ___         ___ _
  / _ \ /\  /\/ __(_)
 / /_\// /_/ / /  | |      GHC Interactive, version 6.3, for Haskell 98.
/ /_\\/ __  / /___| |      http://www.haskell.org/ghc/
\____/\/ /_/\____/|_|      Type :? for help.

Loading package base ... linking ... done.
Loading package altdata ... linking ... done.
Loading package unix ... linking ... done.
Loading package mtl ... linking ... done.
Loading package lang ... linking ... done.
Loading package posix ... linking ... done.
Loading package haskell98 ... linking ... done.
Loading package haskell-src ... linking ... done.
Loading package plugins ... linking ... done.
Loading package eval ... linking ... done.
Prelude> :l Main
Skipping  Main             ( Main.hs, Main.o )
Ok, modules loaded: Main.
Prelude Main> main
Loading package readline ... linking ... done.
           __                          
    ____  / /_  ______ ______          
   / __ \/ / / / / __ `/ ___/     PLugin User's GHCi System, for Haskell 98
  / /_/ / / /_/ / /_/ (__  )      http://www.cse.unsw.edu.au/~dons/hs-plugins
 / .___/_/\__,_/\__, /____/       Type :? for help     
/_/            /____/                  

Loading package base ... linking ... done
plugs> map (\x -> x + 1) [0..10]
[1,2,3,4,5,6,7,8,9,10,11]
plugs> :t "haskell"
"haskell" :: [Char]
plugs> :q
*** Exception: exit: ExitSuccess
Prelude Main> :q
Leaving GHCi.
\end{verbatim}
\end{quote}

\subsection*{Dynamic Typing}

Support is also provided to unwrap and check the type of dynamically
typed plugin values (those wrapper in a \code{toDyn}) via
\code{dynload}. This is the same as \code{load}, except that instead
of a returning the value it finds, it unwraps a dynamically typed
value, checks the type, and returns the unwrapped value. This is to
provide further trust that the symbol you are retrieving from the
plugin is of the type you think it is, beyond that trust you have by
knowing that the plugin was compiled against a shared API. By using
\code{dynload} it is not enough for an object file to just have the
same symbol name as the function you require, it must also carry the
\code{Data.Dynamic} representation of the type, too. \code{pdynload}
rectifies most of \code{dynload}'s limitations, but at the cost of
additional running time.

\section{Compilation Manager Implementation}

Along side the dynamic loader is the compilation manager.  This is a
\code{make}-like system for compiling Haskell source, prior to loading
it. \code{make} checks if a source file is newer than its associated
object file. If so, the source is recompiled to an object file, and a
new dependency file is created, in case the dependencies have changed
in the source. This module can then be loaded.  The idea is to allow
EDSL authors to write plugins without having to touch a compiler: it
is all transparent. It also allows us to enforce type safety in the
plugin by injecting type constraints into the plugin source, as has
been discussed eariler.

The effect is much like \emph{hi} (Hmake Interactive), funnily enough.
An application using both \code{make} and \code{load} behaves like a
Haskell interpreter, using \code{eval}.  You modify your plugin, and
the application notices the change, recompiles it (possibly issuing
type errors) and then reloads the object file, providing the
application with the latest version of the code.
 
An example:
%
\begin{quote}
\scm{
do status <- make "Plugin.hs" []
   obj    <- case status of
                MakeSuccess _ o -> return o
                MakeFailure e   -> mapM_ putStrLn e >> error "failed"

   m_v    <- load obj ["api"] [] "resource"
   v      <- case m_v of
            	LoadSuccess _ v -> return v
            	_               -> error "load failed"
   putStrLn $ field v
}
\end{quote}

\code{make} accepts a source file as an argument, and a (usually empty)
list of GHC flags needed to compile the object file. It then checks to
see if compilation is required, and if so, it calls GHC to compile the
code, with and arguments supplied. If any errors were generated by
GHC, they are returned as the third component of the triple.

Usually it will be necessary to ensure that GHC can find the plugin
API to compile against. This can be done by either making sure the API
is in the same directory as the plugin, or by adding a \code{-i} flag to
\code{make}'s arguments. If the API is created as a ``package'' with a
package.conf file, \code{make} can be given \code{-package-conf} arguments
to the same effect.

Normally, \code{make} generates the \code{.o} and \code{.hi} files in
the same directory as the source file. This is not always desirable,
particularly for interpreter-like applications. To solve this, you can
pass \code{[''-odir'', path]} as elements of the arg list to
\code{make}, and it will respect these arguments, generating the
object and interface file in the directory specified. GHC's argument
\code{''-o''} is also respected in a similar manner, so you could also
say \code{[''-o'', obj]} for the same effect.

\code{make} is entirely optional. All user's have to do to use the
loader on its own is make sure they only load object files that also
have a \code{.hi} file. This will usually be the case if the plugin is
compiled with GHC.

\subsection*{makeWith}

\code{makeWith} merges two source files together, using the function
and value declarations from one file, with any syntax in the second,
creating a new third source file. It then compiles this source file
via \code{make}.

This function exists as a benefit to EDSL authors and is related to
the original motivation for \hsplugins{}, as a .conf file language
library. Configuration files need to be clean and simple, and you
can't rely, or trust, the user to get all the compulsory details
correct. So the solution is to factor out any compulsory syntax, such
as module names, imports, and also to provide a default instance of
the API, and store this code in a separate file provided by the
application writer, not the user.  \code{makeWith} then merges
whatever the user has written, with the syntax stub, generating a
complete Haskell plugin source, with the correct module names and
import declarations. We also ensure the plugin only exports a single
interface value while we are here.

\code{makeWith} thus requires a Haskell parser to parse two source files
and merge the results. We are merging abstract syntax here. This is
implemented using the Language.Haskell parser library. Unfortunately,
this library doesn't implement all of GHC's extensions, so if you wish
to use \code{makeWith} you can only write Haskell source that can be
parsed by this library, which is just H98 and a few extensions. This
is another short coming in the current design that will be overcome
with \code{-package ghc}. Remember, however, for normal uses of
\code{make} and \code{load} you are unrestricted in what Haskell you use.
This is the same restriction present in happy, the Haskell parser,
placed on the code you can provide in the \code{.y} source.

\code{makeWith} also makes use of line pragmas. If the merged file
fails to compile, the judicious use of line number pragmas ensure that
the user receives errors messages reported with reference to their
source file, and not line number in the merged file. This is a
property of the Language.Haskell parser that we can make use of.

An example of \code{makeWith}:
%
\begin{quote}
\scm{
do status <- makeWith "Plugin.in" "Plugin.stub" []
   obj <- case status of
                MakeFailure e   -> mapM_ putStrLn e >> error "failed"
                MakeSuccess _ o -> return o
   m_v  <- load obj [apipath] [] "resource"
   v <- case m_v of
                LoadSuccess _ v -> return v
                _               -> error "load failed"
   putStrLn $ field v
}
\end{quote}

We combine the user's file (\code{Plugin.in}) with a stub of syntax
generating a new, third Haskell file in the default tmpdir. This is
compiled as per usual, producing object and interface files. The
object is then loaded, and we extract the value exported.

Using \code{makeWith} it is possible to write very simple, clear
Haskell plugins, that appear not to be Haskell at all. It is an easy
way to get EDSL user's writing plugins that are actually Haskell
programs, for .e.g, configuration files. See the examples that come
with the src.

\newpage

\section{An Example}

This is an introductory example.

\subsection*{API}

First we need an interface between the application and the plugin.
This module needs to be visible to both the app and the plugin, in the
interest of clear and well-defined interfaces: 
%
\begin{quote}
\scm{
module StringProcAPI (Interface(..), plugin) where

data Interface = Interface {
       stringProcessor :: String -> String
}

plugin :: Interface
plugin = Interface { stringProcessor = id }
}
\end{quote}

Here we define \code{Interface} as the inteface signature for the
object passed between plugin and application. We'll use the record
syntax as it looks intuitive in the plugin. We provide a default
instance, the \code{plugin} value, that can be overwritten in the
actual plugin, ensuring sensible behaviour in the absence of any
plugins. The API should theoretically be compiled with \code{-Onot} to
avoid interface details leaking out into the \code{.hi} file.

\subsection*{The Plugin}

This is our plugin. Note that the plugin will be compiled entirely
seperately from the application. It must only rely on the API, and
nothing in the application source.
%
\begin{quote}
\scm{
module StringProcPlugin (resource) where

import StringProcAPI (plugin)

resource = plugin {
     stringProcessor = reverse 
}
}
\end{quote}

Using the record syntax we overwrite the \code{function} field with our
own value, \code{reverse}. The value \code{resource} is the magic symbol
that must be defined, and which the application will use to find the
data the plugin exports.

Now, we can make this even easier on the plugin writer by the use of a
``stub'' file. \code{makeWith} lets you merge a plugin source with
another Haskell file, and compiles the result into the actual plugin
object. So the application can provide a stub file containing module
declarations and imports, and a default plugin value. Here is an
application-provided stub, factoring out compulsory syntax and type
declarations from the plugin:
%
\begin{quote}
\scm{
module StringProcPlugin ( resource ) where

import StringProcAPI

resource :: Interface
resource = plugin
}
\end{quote}

By factoring out compulsory syntax, the plugin author only has to
provide an overriding instance of the \code{resource} field. So all
the plugin actually consists of, is:
%
\begin{quote}
\scm{
resource = plugin {
      stringProcessor = reverse
}
}
\end{quote}

That is all the code we need! This file may be called anything at all.

More complex APIs may have more fields, of course. The nice thing
about this arrangement is that the user will write some simple syntax,
which will nonetheless by typechecked safely against the API. Errors
are also reported using line numbers from the source file, not the
stub, which makes things less confusing.

\subsection*{The Application}

Now we need to write an application that can use values of the kind
defined in the API, and which can compile and load plugins. The basic
mechanism to compile and load a plugin is as follows:
%
\begin{quote}
\scm{
do status <- make "StringProcPlugin.hs" []
   obj    <- case status of
                MakeSuccess _ o -> return o
                MakeFailure e   -> mapM_ putStrLn e >> error "failed"

   m_v    <- load obj ["."] [] "resource"
   val    <- case m_v of
                LoadSuccess _ v -> return v
                _               -> error "load failed"
}
\end{quote}
%
This code calls \code{make} to compile the plugin source, yielding
wrapper around a handle to an object file. The object can then be loaded
using \code{load}, and the code associated with the symbol
\code{resource} is retrieved. 

We embed this code in a simple shell-like loop, applying the function
exported by the plugin:
%
\begin{quote}
\scm{
import System.Plugins
import StringProcessorAPI
import System.Console.Readline
import System.Exit

source = "Plugin.hs"
stub   = "Plugin.stub"
symbol = "resource"

main = do s <- makeWith source stub []
          o <- case s of
                MakeSuccess _ obj -> do
                        ls <- load obj ["."] [] symbol
                        case ls of LoadSuccess m v -> return (m,v)
                                   LoadFailure err -> error "load failed"
                MakeFailure e -> mapM_ putStrLn e >> error "compile failed"
          shell o

shell o@(m,plugin) = do 
        s <- readline "> "
        cmd <- case s of 
                Nothing          -> exitWith ExitSuccess
                Just (':':'q':_) -> exitWith ExitSuccess
                Just s           -> addHistory s >> return s

        s  <- makeWith source stub []   -- maybe recompile the source
        o' <- case s of
                MakeSuccess ReComp o -> do 
                        ls <- reload m symbol
                        case ls of LoadSuccess m' v' -> return (m',v')
                                   LoadFailure err   -> error "reload failed"
                MakeSuccess NotReq _ -> return o
                MakeFailure e -> mapM_ putStrLn e >> shell o
        eval cmd o'
        shell o'

eval ":?" _ = putStrLn ":?\n:q\n<string>"

eval s (_,plugin) = let fn = (stringProcessor plugin) in putStrLn (fn s)
}
\end{quote}

We have to import the hs-plugins library, and the API. The main loop
proceeds by compiling and loading the plugin for the first time, and
then calls \code{shell}, the interpeter loop. This loop lets us apply
the function in the plugin to strings we supply.  We have to pass
around the \code{(Module, a)} pair we get back from \code{reload}, so
that we can pass it to \code{eval} to do the real work.  The first
\code{eval} case is where we use the record syntax to select the
\code{function} field out of \code{v}, the plugin interface object,
and we apply it to s. Try it out:
%
\begin{quote}
\begin{verbatim}
paprika$ ./a.out 
Loading package base ... linking ... done
Loading objects API Plugin ... done
> :?
":?"
":q"
"<string>"
> abcdefg
gfedcba
\end{verbatim}
\end{quote}

Now, if we edit the plugin while the shell is running, the next time
we type something at the prompt the plugin will be unloaded,
recompiled and reloaded. Because the plugin is really an EDSL, we can
use any Haskell we want, so we'll change the plugin to:
%
\begin{quote}
\scm{
import Data.Char

resource = plugin {
	stringProcessor = my_fn
}

my_fn s = map toUpper (reverse s)
}
\end{quote}

Back to the shell:
%
\begin{quote}
\begin{verbatim}
> abcdefg 
Compiling plugin ... done
Reloading Plugin ... done
GFEDCBA
\end{verbatim}
\end{quote}

And that's it: dynamically recompiled and reload Haskell code!

\section{Multiple Plugins}

It is quite easy to load multiple plugins, that all implement the
common plugin API, and that all export the same value (though
implemented differently). This make \hsplugins{} suitable for
applications that wish to allow an arbitrary number of plugins. The
main problem with multiple plugins is that they may share
dependencies, and if \code{load} na\"ively loaded all dependencies
found in the set of \code{.hi} files associated with all the plugins,
the GHC rts would crash. To solve this the \hsplugins{} dynamic loader
maintains state storing a list of what modules and packages have been
loaded already. If \code{load} is called on a module that is already
loaded, or dependencies are attempted to load, that have already been
loaded, the dynamic loader ignores these extra dependencies. This
makes it quite easy to write an application that will allows an
arbitrary number of plugins to be loaded. An example follows.

\subsection*{Definition}

First we need to define the API that a plugin must type check against,
in order to be valid.  
%
\begin{quote}
\scm{
module API where

data Interface = Interface { 
	valueOf :: String -> String 
}

plugin :: Interface 
plugin = Interface { valueOf = id }
}
\end{quote}

We can then implement a number of plugins that provide values of type
"Interface". We show three plugins that export string manipulation functions:
%
\begin{quote}
\scm{
module Plugin1 where

import API
import Data.Char

resource = plugin {
	valueOf = \s -> map toUpper s
}
}
\end{quote}

\begin{quote}
\scm{
module Plugin2 where

import API
import Data.Char

resource = plugin {
	valueOf = \s -> map toLower s
}
}
\end{quote}

\begin{quote}
\scm{
module Plugin3 where

import API

resource = plugin {
	valueOf = reverse
}
}
\end{quote}

And finally we need to write an application that would use these
plugins. Remember that the application is written without knowledge of
the plugins, and the plugins are written without knowledge of the
application. They are each implemented only in terms of the API, a
shared module and \code{.hi} file. An application needs to make the
API interface available to plugin authors, by distributing the API
object file and \code{.hi} file with the application.
%
\begin{quote}
\scm{
import System.Plugins
import API

main = do
 let plist = ["Plugin1.o", "Plugin2.o", "Plugin3.o"]
 plugins <- mapM (\p -> load p ["."] [] "resource") plist
 let functions = map (valueOf . fromLoadSuc) plugins
 mapM_ (\f -> putStrLn $ f "haskell is for hackers") functions

fromLoadSuc (LoadFailure _)   = error "load failed"
fromLoadSuc (LoadSuccess _ v) = v

}
\end{quote}

This application simply loads all the plugins and retrieves the
functions they export. It then applies each of these functions to a
string, printing the result. We assume for this example that the
plugins are compiled once only, and are not compiled dynamically via
\code{make}. This implies that you have to use \code{GHC} to generate
the \code{.hi} file for each plugin. A sample Makefile to compile the
plugins, and the api:
%
\begin{quote}
\begin{verbatim}
all:
    ghc -Onot -c API.hs
    ghc -O -c Plugin1.hs
    ghc -O -c Plugin2.hs
    ghc -O -c Plugin3.hs
\end{verbatim}
\end{quote}

Ghc creates \code{.hi} files for each plugin, which can be inspected
using the \code{Plugins.BinIface.readBinIface} function. It parses the
\code{.hi} file, generating, roughly, the following:
%
\begin{quote}
\begin{verbatim}
interface "Main" Main
module dependencies: A, B
package dependencies: base, haskell98, lang, unix
\end{verbatim}
\end{quote}

which says that the plugin depends upon a variety of system packages,
and the modules A and B. All these dependencies must be loaded before
the plugin itself.

You then need to compile the application against the API, and against
the \hsplugins{} library:
%
\begin{quote}
\begin{verbatim}
ghc -O --make -package plugins Main.hs
\end{verbatim}
\end{quote}

Running the application produces the following result. Note that the
verbose output can be switched off by compiling \hsplugins{} without
the \code{-DDEBUG} flag. If you look at the \code{.hi} file, using
\code{ghc --show-iface}, you'll see that they all depend on the base
package, and on the API, but the state stored in the dynamic loader
ensures that these shared modules are only loaded once:
%
\begin{quote}
\begin{verbatim}
Loading package base ... linking ... done
Loading object API Plugin1 ... done
Loading object Plugin2 ... done
Loading object Plugin3 ... done

HASKELL IS FOR HACKERS
haskell is for hackers
srekcah rof si lleksah
\end{verbatim}
\end{quote}

Archives of plugins can be loaded in one go if they have been linked
into a .o GHCi package, see \code{loadPackage}.

\newpage

\appendix

\section{License} 

This library is distributed under the terms of the LGPL:

\begin{quote}

Copyright 2003-5, Don Stewart - \url{http://www.cse.unsw.edu.au/~dons}

This library is free software; you can redistribute it and/or
modify it under the terms of the GNU Lesser General Public
License as published by the Free Software Foundation; either
version 2.1 of the License, or (at your option) any later version.

This library is distributed in the hope that it will be useful,
but WITHOUT ANY WARRANTY; without even the implied warranty of
MERCHANTABILITY or FITNESS FOR A PARTICULAR PURPOSE.  See the GNU
Lesser General Public License for more details.

You should have received a copy of the GNU Lesser General Public
License along with this library; if not, write to the Free Software
Foundation, Inc., 59 Temple Place, Suite 330, Boston, MA  02111-1307
USA

\end{quote}

\section{Portability}

The library tries to be portable. There are two major points that
limit easy portabilty. The main issue is a dependence on the GHC dynamic
linker. \hsplugins{} is thus limited to platforms to which GHC's dyn
linker has been ported (this is essentially the same as the platforms
that can run GHCi).

\newpage

\section{A Haskell Interpreter using Plugins}
% \label{sec:plugs}

Here is a full length example of a Haskell interpreter/compiler in the
style of Malcolm Wallace's \code{hi}. Rather than compiling the
user's code to an executable, we use \hsplugins{} to instead load an
object file and execute that instead, using the \code{eval} interface.
This cuts out the linking phase from the process, making turnaround at
the prompt around twice as fast as \code{hi}.

\subsection*{Source of Plugs}

\begin{quote}
\scm{
import System.Eval.Haskell
import System.Plugins

import System.Exit              ( ExitCode(..), exitWith )
import System.IO
import System.Console.Readline  ( readline, addHistory )

symbol = "resource"

main = do
        putStrLn banner 
        putStr "Loading package base" >> hFlush stdout
        loadPackage "base" 
        putStr " ... linking ... " >> hFlush stdout
        resolveObjs
        putStrLn "done"

        shell []

shell :: [String] -> IO ()
shell imps = do 
        s <- readline "plugs> "
        cmd <- case s of 
                Nothing          -> exitWith ExitSuccess
                Just (':':'q':_) -> exitWith ExitSuccess
                Just s           -> addHistory s >> return s
        imps' <- run cmd imps
        shell imps'

run :: String -> [String] -> IO [String]
run ""   is = return is
run ":?" is = putStrLn help >> return is

run ":l" _             = return []
run (':':'l':' ':m) is = return (m:is)

run (':':'t':' ':s) is = do 
        ty <- typeOf s is
        when (not $ null ty) (putStrLn $ s ++ " :: " ++ ty)
        return is

run (':':_) is = putStrLn help >> return is

run s is = do 
        s <- unsafeEval ("show $ "++s) is
        when (isJust s) (putStrLn (fromJust s))
        return is

banner = "\ 
\           __                          \n\
\    ____  / /_  ______ ______          \n\
\   / __ \\/ / / / / __ `/ ___/     PLugin User's GHCi System, for Haskell 98\n\
\  / /_/ / / /_/ / /_/ (__  )      http://www.cse.unsw.edu.au/~dons/hs-plugins\n\
\ / .___/_/\\__,_/\\__, /____/       Type :? for help     \n\
\/_/            /____/                  \n"

help = "\
\Commands :\n\ 
\  <expr>               evaluate expression\n\ 
\  :t <expr>            show type of expression (monomorphic only)\n\ 
\  :l module            bring module in to scope\n\ 
\  :l                   clear module list\n\ 
\  :quit                quit\n\ 
\  :?                   display this list of commands"
}
\end{quote}

\subsection*{A Transcript}

And a transcript:
%
\begin{quote}
\begin{verbatim}
$ ./plugs
           __                          
    ____  / /_  ______ ______          
   / __ \/ / / / / __ `/ ___/     PLugin User's GHCi System, for Haskell 98
  / /_/ / / /_/ / /_/ (__  )      http://www.cse.unsw.edu.au/~dons/hs-plugins
 / .___/_/\__,_/\__, /____/       Type :? for help     
/_/            /____/                  

Loading package base ... linking ... done
plugs> 1
1
plugs> let x = 1 + 2 in x
3
plugs> :l Data.List
plugs> case [1,3,2] of x -> sort x
[1,2,3]
plugs> reverse [1,3,2]
[2,3,1]
plugs> map (\x -> (x,2^x)) [1,2,3,4,5,6,7,8,9,10]
[(1,2),(2,4),(3,8),(4,16),(5,32),(6,64),(7,128),(8,256),(9,512),(10,1024)]
plugs> :t "haskell"
"haskell" :: [Char]
plugs> :quit
\end{verbatim}
\end{quote}

\end{document}
